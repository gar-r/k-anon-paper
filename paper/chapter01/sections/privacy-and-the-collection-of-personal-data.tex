A few years ago in Minneapolis an angry man walked into a retail store called \textit{Target} and demanded to see the manager. \textit{``My daughter got this in the mail!''} he exclaimed, showing some coupons in his hand. \textit{``She is still in high school, and you're sending her coupons for baby clothes and cribs?
Are you trying to encourage her to get pregnant?''}~\cite{youtube01,nytimes01} The coupons were indeed addressed to the man's daughter and contained advertisements for maternity clothing, nursery furniture and pictures of smiling infants.
The manager apologized, and called a few days later to apologize again.

On the phone, though, the father was somewhat abashed. \textit{``I had a talk with my daughter''}.
he said. \textit{``It turns out there has been some activities in my house I haven't been completely aware of.
She's due in August.
I owe you an apology.''}~\cite{nytimes01} But how did the retail store know that the girl is pregnant before her own father?
As a matter of fact \textit{Target} was observing the purchasing habits of its customers through their membership cards.
In this case, a change like purchasing scent-free lotions and soap, extra-big bags of cotton balls, hand sanitizers, washcloths and supplements containing calcium magnesium and zinc indicated a likely pregnancy to the retail store's cleverly constructed advertising algorithm.
In fact, the algorithm was so advanced, that it could even predict the due date with a relatively small time window so that the store can send coupons timed to very specific stages of pregnancy.

In our modern age data is power.
Data is now a commodity --- arguably the world's most valuable commodity, dethroning oil from its former number one position~\cite{economist01}.
Big software companies have recognized this a long time ago and offer convenient, easy to use software and services \textit{``for free}.'' In reality however, nothing is free.
Users of these services are paying with their personal data, ranging from simple personal identifiers to more complex usage statistics and metrics.
This data is most often cross-referenced with other similarly collected databases to draw conclusions and offer a more personalized user experience --- including advertisements.

As a result, we are now subject to a greater level of surveillance than in any point in history, and we hand over most of our data willingly~\cite{factortech01}.
It can be expected, that the increasing amount of \textit{internet-of-things} devices and the introduction of \textit{5G networking} will result in an even bigger surge in the amount and types of data collected.