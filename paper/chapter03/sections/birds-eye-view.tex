In this section we present a high-level overview of how our anonymization library works.
The flowchart on Figure~\ref{fig:birds_eye_view} illustrates what happens to the input data after it is passed into the algorithm.

The terminal states are denoted by rounded rectangles: the anonymization can terminate with an error if the input is not valid or not in the correct format (\emph{error} state).
Otherwise it will terminate by returning the anonymized data table (\emph{done} state).

Processing steps are represented with rectangles, while data in the different stages is represented by diamond shapes.
We can observe, that the input data is first transformed into its graph representation, then it goes through the anonymization process discussed in the previous chapter.
The anonymization process yields partitions.
Finally, the partitions are be mapped back to data rows. Rows in the same partition are generalized until they are equivalent, and the output anonymized data table is constructed.

\vspace{\baselineskip}
In this section we present a high-level overview of how our anonymization library works.
The flowchart on Figure~\ref{fig:birds_eye_view} illustrates what happens to the input data after it is passed into the algorithm.

The terminal states are denoted by rounded rectangles: the anonymization can terminate with an error if the input is not valid or not in the correct format (\emph{error} state).
Otherwise it will terminate by returning the anonymized data table (\emph{done} state).

Processing steps are represented with rectangles, while data in the different stages is represented by diamond shapes.
We can observe, that the input data is first transformed into its graph representation, then it goes through the anonymization process discussed in the previous chapter.
The anonymization process yields partitions.
Finally, the partitions are be mapped back to data rows. Rows in the same partition are generalized until they are equivalent, and the output anonymized data table is constructed.

\vspace{\baselineskip}
In this section we present a high-level overview of how our anonymization library works.
The flowchart on Figure~\ref{fig:birds_eye_view} illustrates what happens to the input data after it is passed into the algorithm.

The terminal states are denoted by rounded rectangles: the anonymization can terminate with an error if the input is not valid or not in the correct format (\emph{error} state).
Otherwise it will terminate by returning the anonymized data table (\emph{done} state).

Processing steps are represented with rectangles, while data in the different stages is represented by diamond shapes.
We can observe, that the input data is first transformed into its graph representation, then it goes through the anonymization process discussed in the previous chapter.
The anonymization process yields partitions.
Finally, the partitions are be mapped back to data rows. Rows in the same partition are generalized until they are equivalent, and the output anonymized data table is constructed.

\vspace{\baselineskip}
In this section we present a high-level overview of how our anonymization library works.
The flowchart on Figure~\ref{fig:birds_eye_view} illustrates what happens to the input data after it is passed into the algorithm.

The terminal states are denoted by rounded rectangles: the anonymization can terminate with an error if the input is not valid or not in the correct format (\emph{error} state).
Otherwise it will terminate by returning the anonymized data table (\emph{done} state).

Processing steps are represented with rectangles, while data in the different stages is represented by diamond shapes.
We can observe, that the input data is first transformed into its graph representation, then it goes through the anonymization process discussed in the previous chapter.
The anonymization process yields partitions.
Finally, the partitions are be mapped back to data rows. Rows in the same partition are generalized until they are equivalent, and the output anonymized data table is constructed.

\vspace{\baselineskip}
\input{chapter03/figures/birds-eye-view.tex}